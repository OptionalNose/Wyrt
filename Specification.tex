\documentclass{article}
\usepackage{bm}
\title{PROJECT CHARLES}
\author{Jacob Beveridge}
\date{v0.1 WIP}
\begin{document}
\newcommand{\ret}{$\bm{-\!\!>}$}

\maketitle
\tableofcontents
\section{About This Document}
	This Document defines PROJECT CHARLES version 0.1 Source Code.
	\subsection{Tooling} The `Compiler' is any program that takes PROJECT CHARLES Source Code as input and produces either an object or executable file as output. An `Error' is a diagnostic message produced by the Compiler upon the input of erroneous Source Code, accompianied by the Compiler not producing output.
	\subsection{Notation} Any text in italics is a formula thast defines syntax. Any non-bolded text in a formula represents some non-fixed text. Any bolded text in a formula represents fixed texts. Unless otherwise stated: two pieces of non-fixed text must be separated by at least one whitespace character, and a piece of fixed next need not but may be separated from any surrounding non-fixed text by any amount of whitespace. Square Brackets (\textit{[]}) surround optional text. An Asterisk(\textit(*)) means that the preceding optional text may be repeated any number of times, including zero.
\section{Definitions}
	\begin{description}
	\item[Module] The contents of a PROJECT CHARLES source file. Comprised of Declarations.
	\item[Declaration] Of the form \textit{access identifier \textbf{:} type [\textbf{=} value]\textbf{;}}. Creates a new Identifier \textit{identifier} which refers to a Value of Type \textit{type} and can be accessed according to Access Modifier \textit{access}. If a Value \textit{value} is present, then the Value of \textit{identifier} is equal to \textit{value} until \textit{identifier} is Modified and we say that \textit{identifier} is `initialized' to \textit{value}. If \textit{type} is a Function Type then \textit{identifier} is a Function and \textit{access} is not \textit{const}. The Compiler shall issue an Error if \textit{value} is not present and \textit{access} is \textit{const}, or if the Type of \textit{value} is not compatible with \textit{type}.
	\item[Access Modifier] \textit{const}.
	\item[Identifier] A sequence of characters starting with any of:
		\begin{quote}
		$\_$ @ A B C D E F G H I J K L M N O P Q R S T U V W X Y Z a b c d e f g h i j k l m n o p q r s t u v w x y z
		\end{quote}
		followed by any number of any of:
		\begin{quote}
		$\_$ @ A B C D E F G H I J K L M N O P Q R S T U V W X Y Z a b c d e f g h i j k l m n o p q r s t u v w x y z 0 1 2 3 4 5 6 7 8 9
		\end{quote}
		The Compiler shall issue a warning if it encounters an Identifier lexicographically identical to a Keyword. Some Identifiers have a special meaning or must be used a certain way; see Behavior, Special Identifiers.
	\item[Type] Either:
		\begin{description}
		\item[Data Type] One of:
			\begin{quote}
			void char u8 u16 u32 u64 u128 u256 i8 i16 i32 i64 i128 i256 f16 f32 f64 f128 
			\end{quote}
			Two Data Types are `equivalent' if and only if they are the same. Two Data Types are `compatible' if and only if they are equivalent.
		\item[Function Type] Of either the form \textit{(void) \ret type} or \textit{(identifier\textbf{:} type [\textbf{,} identifier\textbf{:} type]*) \ret type}, where each \textit{type} is a potentially unique Data Type. In the First Case, the Function Type has no Parameters. In the Second Case, each \textit{identifier} must be lexicographically unique, and is said to be a Parameter of Type \textit{type} of the Function Type. The Compiler shall issue an Error if the Type of any Parameter is \textit{void}. In both cases, the \textit{type} after the \textit{\ret} is the Return Type of the Function Type. Two Function Types are `equivalent' if and only if they have an equal number of Parameters, each matching Parameter has an equivalent Data Type, and both Return Types are equivalent. Two Function Types are `compatible' if and only if they are equivalent.
		\end{description}
		A Data Type and a Function type are never equivalent nor compatible.
	\item[Value] A Block.
	\item[Modification] Change the Value of an Identifier, potentially to an equivalent Value.
	\item[Keyword] An Access Modifier, or any of:
		\begin{quote}
		return
		\end{quote}
	\item[Block] Of the form \textit{\textbf{$\{$} statements \textbf{$\}$}}, where \textit{statements} is any number of Statements. The Type of a Block is compatible with all Function Types, and incompatible with all Data Types.
	\item[Statement] One of:
		\begin{description}
		\item[Return Statement] \hfill \\ Of the form \textit{\textbf{return} value} where \textit{value} is a Value and is called the `Return Value'. The Compiler shall issue an Error if the Type of \textit{value} is not compatible with the Return Type of the Function which is initialized to the enclosing Block.
		\end{description}
		Follwed by a Semicolon(;). 
	\end{description}
\subsection{Special Identifiers}
	\begin{itemize}
	\item[] There are several ``builtin'' Types which are predefined Identifiers; see Definitions, Data Type. The Compiler shall issue an Error if any of these Identifiers are used as anything other than a Data Type.
	\item[] ``main'' must be Declared with a Function Type of ``(void) \ret u8''. 
	\end{itemize}
\section{Behavior}
	\subsection{Entry and Exit} The Program starts at the first line of ``main'' and proceeds statement by statement. The Program terminates when the ``main'' function returns.
	\subsection{Statements}
		\begin{description}
		\item[Return Statement] Upon execution, causes the enclosing function to terminate and evaluate to the Return Value of the Return Statement.
		\end{description}
\end{document}