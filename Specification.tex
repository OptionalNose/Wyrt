\documentclass{article}
\usepackage{bm}
\usepackage[T1]{fontenc}
\title{Wyrt}
\author{Jacob Beveridge}
\date{v0.1 WIP}
\begin{document}

\maketitle
\tableofcontents
\section{About This Document}
	This Document defines Wyrt version 0.1 Source Code.
	\subsection{Tooling} The `Compiler' is any program that takes Wyrt Source Code as input and produces either an object or executable file as output. An `Error' is a diagnostic message produced by the Compiler upon the input of erroneous Source Code, accompianied by the Compiler not producing output.
	\subsection{Notation} Any text in italics is a formula that defines syntax. Any non-bolded text in a formula represents some non-fixed text. Any bolded text in a formula represents fixed text. Unless otherwise stated: two pieces of non-fixed text must be separated by at least one whitespace character, and a piece of fixed next need not but may be separated from any surrounding non-fixed text by any amount of whitespace. Square Brackets (\textit{[]}) surround optional text. An Asterisk(\textit*) means that the preceding optional text may be repeated any number of times, including zero (Trailing Commas are omitted). A pipe (\textit{|}) represents an either-or option. 
\section{Definitions}
	\begin{description}
	\item[Module] The contents of a Wyrt source file. Comprised of Function Definitions.
		
	\item[Function Definition] Of the form \textit{\textbf{fn} name \textbf ( [ argname: argtype, \textbf ] ) returntype \textbf \{ [statement]* \textbf \} } \\
		Where `name' is an identifier and the name of the function, `argname' and `argtype' are the name (identifier) and type of the argument respectively, `returntype' is the type returned by the function, and `statement' is a statement.

	\item[Identifier] A sequence of characters starting with any of:
		\begin{quote}
		$\_$ @ A B C D E F G H I J K L M N O P Q R S T U V W X Y Z a b c d e f g h i j k l m n o p q r s t u v w x y z
		\end{quote}
		followed by any number of any of:
		\begin{quote}
		$\_$ @ A B C D E F G H I J K L M N O P Q R S T U V W X Y Z a b c d e f g h i j k l m n o p q r s t u v w x y z 0 1 2 3 4 5 6 7 8 9
		\end{quote}
		The Compiler shall issue an error if it encounters an Identifier lexicographically identical to a Keyword. Some Identifiers have a special meaning or must be used a certain way; see Behavior, Special Identifiers.

	\item[Type] One of
		\begin{description}
		\item[Primitive] One of:
			\begin{quote}
			void u8 u16 u32 u64
			\end{quote}
			Where `void' is a zero-sized object.
				`u8', `u16', `u32', and `u64' (the integer types) are unsigned integers of 8, 16, 32, and 64 bits respectively.

		\item[Pointer] \textit{\textbf \& access type}
			Represents the address of an object of type `type', able to be accessed according to access-modifier `access'.

		\item[Array] \textit{\textbf [ intlit \textbf | $\_$ ] type}
			Represents `intlit' number of objects of type `type'. If an underscore is used, the length of the array is implied from its initializer, which must be present.

		\item[Slice] \textit{\textbf{[]} access type}
			Represents an object comprised of a length `len' (u64) and a pointer to a `len' number of objects of type `type' able to be accessed by `access'.

		\end{description}
			Any Data Type that will implicitly cast into another Data Type is said to be compatible with that Data Type.
			Any `var'-qualified Data Type (pointer or slice) will implicitly cast to an equivalent const-qualified or abyss-qualified Data Type. 
			Any integer type will implicitly cast to a larger integer type of the same signedness.
	\item[Access Modifier] \hfill \\
		\begin{description}
		\item[const] Read-Only. The Compiler shall issue an error if it encounters an attempted write to a constant variable, pointer, or slice.
		\item[abyss] Write-Only. The Compiler shall issue an error if it encounters an attempted read from an abyssal variable, pointer, or slice.
		\item[var] Read-Write. Any `var'-qualified value will implicitly cast to a `const'- or `abyss'-qualified value.
		\end{description}

	\item[Keyword] An Access Modifier, Primitive Type, or any of:
		\begin{quote}
		fn return
		\end{quote}

	\item[Statement] One of:
		\begin{description}
		\item[Return Statement] \hfill \textit{\textbf{return} value} \\
			Where \textit{value} is a Value and is called the `Return Value'.
			The Compiler shall issue an Error if the Type of \textit{value} is not compatible with the Return Type of the Function Definition in which this statement is located.

		\item[Variable Declaration] \hfill \textit{access name \textbf : type [$=$ value]} \\
			Where `access' is an access modifier, `name' is an identifier, and `value' is a value.
			If `access' is `const', then the Compiler will issue an error if `value' is not present.
			Declares a Variable named `name', and if present, initializes its value to `value'.

		\item[Assignment]
			\hfill 1. \textit{lvalue \textbf = value} \\
			{}\hfill 2. \textit{lvalue \textbf += value} \\
			{}\hfill 3. \textit{lvalue \textbf -= value} \\
			{}\hfill 4. \textit{lvalue \textbf *= value} \\
			{}\hfill 5. \textit{lvalue \textbf /= value} \\
			(1) Assigns Lvalue `lvalue' to `value'. The Compiler will issue an error if `lvalue' is `const'-qualified. \\
			(2) Assigns Lvalue `lvalue' to 'lvalue $+$ value'. The Compiler will issue an error if `lvalue' is not `const'-qualified.\\
			(3) Assigns Lvalue `lvalue' to 'lvalue $-$ value'. The Compiler will issue an error if `lvalue' is not `const'-qualified.\\
			(4) Assigns Lvalue `lvalue' to 'lvalue $*$ value'. The Compiler will issue an error if `lvalue' is not `const'-qualified.\\
			(5) Assigns Lvalue `lvalue' to 'lvalue $/$ value'. The Compiler will issue an error if `lvalue' is not `const'-qualified.\\

		\item[Function Call] \hfill \textit{fnname \textbf ( [value,]* \textbf ) } \\
			Passes each `value' as an argument to the function `fnname' and executes it. The Compiler will issue an error if the type of each `value' is not compatible with the type of the argument to which it is passed.

		\end{description}
		Follwed by a Semicolon(;). 

	\item[Value]
		One of:
		\begin{description}
		\item[Integer Literal] An integer.
		\item[Identifier] The value of the variable `identifier'. The Compiler will issue an error if the variable is `abyss'-qualified.
		\item[Subscript] \textit{ident\textbf[ value \textbf]} The value of the `value'th object from the start of the array or slice `ident'.
			The Compiler will issue an error if `ident' is `abyss'-qualified.
			The Compiler will issue an error if the type of `value' is not an integer type.

		\item[Expression]
			One of:
			\begin{description}
			\item[\textit{lhs \textbf + rhs}] Adds value `lhs' to value `rhs'
			\item[\textit{lhs \textbf - rhs}] Subtracts value `rhs' from value `lhs'
			\item[\textit{lhs \textbf * rhs}] Multiplies value `lhs' by value `rhs'
			\item[\textit{lhs \textbf / rhs}] Divides value `lhs' by value `rhs'. On integer types the result is truncated towards zero.
			\item[\textbf ( \textit{value} \textbf )] The Value of `value'.
			\end{description}
			Expressions are parsed according to the rules of Operator Precedence.

		\item[Address] \textbf \& \textit{ident} \\
			The Address the variable `ident' is stored in. If `ident' is an array, this value is a slice whose length is that of the array.
			Else, this value is a pointer.
			The resulting value has the same access-qualification as `ident'.

		\item[Dereference] \textbf * \textit{value} \\
			The Value at the address `value'. The Compiler will issue an error if the type of `value' is not a pointer type, or is an `abyss'-qualified pointer type.

		\end{description}

	\item[Lvalue] \hfill \\
		\begin{description}
		\item[Identifier] The value of the variable `identifier'. The Compiler will issue an error if the variable is `const'-qualified.

		\item[Subscript] \textit{arr\textbf[ index \textbf]} \\
			The value of the `index'th object from the start of the array or slice `arr'.
			The Compiler will issue an error if `arr' is `const'-qualified.
			The Compiler will issue an error if the type of 'index' is not an integer type.

		\item[Dereference] \textbf * \textit{value} \\
			The Value at the address of `value'. The Compiler will issue an error if the type of `value' is not a pointer type, or is a `const'-qualified pointer type.

		\end{description}
	\item[Operator Precedence]
		Expressions are parsed according to the following list. Operators in the categories higher up are parsed before operators in lower categories.
		Operators in the same category are parsed left-to-right (left-associative).
		\begin{itemize}
		\item $[]$ (subscript)
		\item \& $*$ (unary dereference)
		\item $/$ $*$ (binary multiplication)
		\item $+$ $-$
		\end{itemize}
	\end{description}

\section{Behavior}
	\subsection{Entry and Exit} The Program starts by executing the function ``main''. The Program terminates when the ``main'' function returns.
		The ``main'' function must have take zero arguments and return a u8.
\end{document}
